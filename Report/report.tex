%%%%%%%%%%%%%%%%%%%%%%%%%%%%%%%%%%%%%%%%%
% Short Sectioned Assignment
% LaTeX Template
% Version 1.0 (5/5/12)
%
% This template has been downloaded from:
% http://www.LaTeXTemplates.com
%
% Original author:
% Frits Wenneker (http://www.howtotex.com)
%
% License:
% CC BY-NC-SA 3.0 (http://creativecommons.org/licenses/by-nc-sa/3.0/)
%
%%%%%%%%%%%%%%%%%%%%%%%%%%%%%%%%%%%%%%%%%

%----------------------------------------------------------------------------------------
%	PACKAGES AND OTHER DOCUMENT CONFIGURATIONS
%----------------------------------------------------------------------------------------

\documentclass[paper=a4, fontsize=11pt]{scrartcl} % A4 paper and 11pt font size

\usepackage[T1]{fontenc} % Use 8-bit encoding that has 256 glyphs
\usepackage{fourier} % Use the Adobe Utopia font for the document - comment this line to return to the LaTeX default
\usepackage[english]{babel} % English language/hyphenation
\usepackage{amsmath,amsfonts,amsthm} % Math packages
\usepackage{lipsum} % Used for inserting dummy 'Lorem ipsum' text into the template
\usepackage{placeins}
\usepackage{graphicx}
\usepackage{sectsty} % Allows customizing section commands
\allsectionsfont{\centering \normalfont\scshape} % Make all sections centered, the default font and small caps

\usepackage{fancyhdr} % Custom headers and footers
\pagestyle{fancyplain} % Makes all pages in the document conform to the custom headers and footers
\fancyhead{} % No page header - if you want one, create it in the same way as the footers below
\fancyfoot[L]{} % Empty left footer
\fancyfoot[C]{} % Empty center footer
\fancyfoot[R]{\thepage} % Page numbering for right footer
\renewcommand{\headrulewidth}{0pt} % Remove header underlines
\renewcommand{\footrulewidth}{0pt} % Remove footer underlines
\setlength{\headheight}{13.6pt} % Customize the height of the header

\numberwithin{equation}{section} % Number equations within sections (i.e. 1.1, 1.2, 2.1, 2.2 instead of 1, 2, 3, 4)
\numberwithin{figure}{section} % Number figures within sections (i.e. 1.1, 1.2, 2.1, 2.2 instead of 1, 2, 3, 4)
\numberwithin{table}{section} % Number tables within sections (i.e. 1.1, 1.2, 2.1, 2.2 instead of 1, 2, 3, 4)

\setlength\parindent{0pt} % Removes all indentation from paragraphs - comment this line for an assignment with lots of text

%----------------------------------------------------------------------------------------
%	TITLE SECTION
%----------------------------------------------------------------------------------------

\newcommand{\horrule}[1]{\rule{\linewidth}{#1}} % Create horizontal rule command with 1 argument of height

\title{	
\normalfont \normalsize 
\textsc{University of Crete, Computer Science Department} \\ [25pt] % Your university, school and/or department name(s)
\horrule{0.5pt} \\[0.4cm] % Thin top horizontal rule
\huge CS-527 Parallel Computer Architecture \\ 
Pagerank Multithreading Algorithm % The assignment title
\horrule{2pt} \\[0.5cm] % Thick bottom horizontal rule
}

\author{Iacovos Kolokasis \\(kolokasis@csd.uoc.gr) \\AM:1039} % Your name

\date{\normalsize\today} % Today's date or a custom date

\begin{document}

\maketitle % Print the title

%----------------------------------------------------------------------------------------
%	Abstract
%----------------------------------------------------------------------------------------

\section{Abstract}

PageRank is a computation on directed graphs which iteratively updates a rank
for each vertex. This is a good feature to know the quality of each node (e.g.
in social networks find out the most social people, on the webpages use by
Google to bring the most social webpages first as a result of a search.) This is
a problam has a lot of computation accross the vertices grows up. There are are
many implementations of this algorithm. So in this project I implement Pagerank
in multithreading version comparing the running time versus the Spark GraphX
implementations. % Put something here about the results %


%----------------------------------------------------------------------------------------
%	Introduction
%----------------------------------------------------------------------------------------

\section{Introduction}

We are going through the era of Big Data and the massive computations. All the
publish work focus on the scalability, which is the most important of a
distributed platform. The issue of performance is not take into account and
sometimes sucrifices in order to achieve the scalability.

In this work we try to measure the overheads from this distributed platforms.
In order to achieve this target, we get the PageRank algorithm. Implement a
multitheading version on shared memmory and try to compare the run time
execution with the available implemented version on Spark GraphX. The
implementation of the GraphX is th


%----------------------------------------------------------------------------------------
%       Methodolodgy
%----------------------------------------------------------------------------------------
\subsection{Pagerank}
Pagerank is a computation on directed graphs wich iteratively updates a rank
maintained for each vertex. In each iteration a vertex rank is uniformly
devided among it's outgoing neighbors and then set to be the accumulated of
scaled rank from incoming neighbors. A dumby fuctor alpha is applied to the
ranks.

Initialize all the node with a default probability: 
$$PR(p_i;0) = \frac{1}{N}$$
                
For every node in the graph calculate a rank on every iteration: 
$$PR(p_i;{t+1}) = \frac{1-d}{N} + d\sum_{p_j\inM(p_i)}\frac{PR(p_j)}{out(p_j)}$$ 

\subsection{Multithreading PageRank}

The implementation of the algorithm is written in C language and use the PThread
Library. At the begining the algorithm read an input edge list file which
represent the graph. At this step is create the graph as an adjency matrix in
memory and keep all the informations about each vertex neigbor (in neighbors and
out neighbors). At the same time the application initialize evety vertex of the
graph with an initial probability. 

Then we pass to the phase, where the graph is ready and we will start the
PageRank executions. We select to run the algorithm for 10 fixed iteration
steps. At every step we create a number of threads as we want and assign to
every thread to compute the Pagerank of some nodes. We assign every vertex to
run on a different processor in order to achieve better parallelization.

In the next phase, after the threads execution finish we update all the old
probabilities with the new one. And then continue along to finish all the
iterations.

At the end, after the computation complete we get an output file in the format
of <node Id> <Rank>. 

\subsection{GraphX PageRank}
GraphX is a Spark API for graphs and distributed graph parallel computations.
GraphX is extends the Spark RDD abstraction by introducing the Resiliant
Distributed Property Graph, which is a directed multigraph with properties
attached to each vertex and edge. The vertex exchange messages between them
locally and across the clusters to perform different computations.

At first GraphX read an input graph dataset of the format <src> <dst>. Where is
a directed graph from source Id to the destination Id. Construct the graph and
assign a partition of the graph to each worker. The we are ready to start the
computation of the pagerank algorithm.

GraphX Pagerank implementations use Pregel interface and run the algorithm for a
fixed number of iterations. At the start initialize the pagerankGraph with each
edge attribute having weight 1/outDegree and each vertex with attribute 1.0.
Then compute the outgoing rank contributions of each vertex, by performing local
preaggregation, and do the final aggregation at the receiving vertices.  After
that require a shuffle for aggregation. Apply the final rank updates to get the
new ranks, using join to preserve ranks of vertices that didn't receive a
message.  Requires a shuffle for broadcasting updated ranks to the edge
partitions.

%----------------------------------------------------------------------------------------
%	Experimental Setup
%----------------------------------------------------------------------------------------
\section{Experimental Setup}
In order to perform our experiments we generate two datasets as shown in
\ref{tab:dataset}. We run our experiment on a machine with Intel(R) Xeon(R) CPU
E5-2630 V3 @ 2.40GHz CPU, 32 number of cores, 256GB total memory with CentOS
Linux 7 (Core) OS. 
About multithreading implementation of Pagerank we test it
using 1, 2, 4, 8, 16, 32 number of threads respectively executing each thread on
a different core (one thread per core). According to the GraphX Pagerank
Implementation we run two different experiments on a single node in order to be
comparable with the multithreading implementation. 


\begin{table}
\centering
    \begin{tabular}{|l|c|c|c|}
        \hline
        \textbf{Dataset} &\textbf{Vertex} &\textbf{Edges} \\
        \hline
        Dataset1  &10,000  &10,000,000 \\
        \hline
        Dataset2   &100,000  &20,000,000 \\
        \hline
    \end{tabular}
    \caption{Input Datasets}
    \label{tab:dataset}
\end{table}

%----------------------------------------------------------------------------------------
%	Results
%----------------------------------------------------------------------------------------

%----------------------------------------------------------------------------------------
%	Conclussions
%----------------------------------------------------------------------------------------













\end{document}
